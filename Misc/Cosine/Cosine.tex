\documentclass{article}
\usepackage[utf8]{inputenc}
\usepackage{amsmath}

\title{Cosine Identities}
\author{Cecilia}
\date{January 2023}

\begin{document}
\maketitle
The goal of this document is to prove a few product identities for the cosine function. The key tools we will use are the Chebyshev polynomials.

\section{Know our tools}
The Chebyshev polynomials are a sequence of polynomials, it has two kinds. The first kind is denoted as $ T_n(x) $ and the second kind is denoted as $ U_n(x) $.

\begin{enumerate}
    \item The roots of $ T_n(x) $ are $ \cos\left(\frac{2k + 1}{2n}\pi \right) $ for $ k \in 0 \cdots n - 1 $
    \item The roots of $ U_n(x) $ are $ \cos\left(\frac{k}{n + 1}\pi \right) $ for $ k \in 1 \cdots n $
    \item The constant term of a $ T_{2n} $ is $ -1 $.
    \item The linear term of a $ T_{2n-1} $ is $ (-1)^{n+1} (2n-1)x $.
    \item The constant term of a $ U_{2n} $ is $ (-1)^n $.
    \item The linear term of a $ U_{2n-1} $ is $ (-1)^{n+1} 2nx $.
    
    \item The leading coefficient of $ T_{n} $ is $ 2^{n-1} $
    \item The leading coefficient of $ U_{n} $ is $ 2^n $
\end{enumerate}

All these properties and their proofs are available publicly.

\section{Problem 1}
Our goal is to solve this:
\begin{eqnarray*}
  \prod_{i=1}^{m}{\cos\left(\frac{2k-1}{4m}\pi\right)} \\
\end{eqnarray*}
Note that these numbers are exactly the first half of the roots of $ T_{2m}(x) $. Naturally, we would like to explore if it is possible to get some symmetry between the roots. And there is an easy one.

\begin{eqnarray*}
  \cos(\pi - x) &=& -\cos(x) \\
\end{eqnarray*}

Geometrically, the roots of the Chebyshev polynomial is simply dividing $ \pi $ by $ 2n $ pieces (so you have $ 2n + 1 $ angle, including 0 and $ \pi $, and then pick the one with odd index. So it is easy to see for even $ n $, there will be exactly of them that are negative, and the symmetry above tell us the magnitude is the same. 

Therefore the answer of the product can be found using the product of roots.

The product of roots of $ T_{2m} $ is $ \frac{-1}{2^{2m-1}} $. Therefore the answer is $ \frac{\sqrt{2}}{2^m} $

\section{Problem 2}
Our goal is to solve this:
\begin{eqnarray*}
  \prod_{i=1}^{m-1}{\cos\left(\frac{k}{2m}\pi\right)} \\
\end{eqnarray*}
Note that these numbers are exactly the first half of the roots of $ U_{2m-1}(x) $. Obviously we cannot include $ k = m $ because then that would be a $ \cos\left(\frac{\pi}{2}\right) = 0 $ and the product is 0, and for $ m + 1 \le k \le 2m - 1$, we have the same symmetry as above.

Therefore the answer of the product can be found using the product of roots, with the caveat that we need to divide $ x $ to get rid of the 0 root.

The product of roots of $ \frac{U_{2m-1}}{x} $ is $ \frac{(-1)^{m+1} 2m}{2^{2m-1}} $. Therefore the answer is $ \frac{\sqrt{m}}{2^{m-1}} $.

\section{Problem 3}
Our goal is to solve this:
\begin{eqnarray*}
  \prod_{i=1}^{m}{\cos\left(\frac{k}{2m + 1}\pi\right)} \\
\end{eqnarray*}
Note that these numbers are exactly the first half of the roots of $ U_{2m}(x) $. For $ m + 1 \le k \le 2m - 1$, we have the same symmetry as above.

Therefore the answer of the product can be found using the product of roots.

The product of roots of $ U_{2m} $ is $ \frac{(-1)^m}{2^{2m}} $. Therefore the answer is $ \frac{1}{2^m} $.

\section{Problem 4}
Our goal is to solve this:
\begin{eqnarray*}
  \prod_{i=1}^{m}{\cos\left(\frac{2k - 1}{4m + 2}\pi\right)} \\
\end{eqnarray*}
Note that these numbers are exactly the first half of the roots of $ T_{2m+1}(x) $. Obviously we cannot include $ k = m + 1 $ because then that would be a $ \cos\left(\frac{\pi}{2}\right) = 0 $ and the product is 0. For $ m + 2 \le k \le 2m + 1 $, we have the same symmetry as above.

Therefore the answer of the product can be found using the product of roots, with the caveat that we need to divide $ x $ to get rid of the 0 root.

The product of roots of $ \frac{T_{2m+1}}{x} $ is $ \frac{(-1)^{2m+1} (2m+1)}{2^{2m}} $. Therefore the answer is $ \frac{\sqrt{2m+1}}{2^m} $
\end{document}
