\documentclass{article}
\usepackage[utf8]{inputenc}
\usepackage{hyperref}

\title{PDE}
\author{Cecilia Chan}
\date{April 2021}

\begin{document}

\maketitle

\section*{Problem}
The goal of this document is to solve this PDE using Fourier transform.
\begin{eqnarray*}
      u_t &=& ku_{xx} \\
  u(0, x) &=& \phi(x)
\end{eqnarray*}

\section*{Fourier transform}
Different authors defines Fourier transform differently, to avoid confusion. In this document, we will adopt this particular definition. In this section, we will derive the necessary results.

The Fourier transform of $ f(x) $ to defined to be $ F(\omega) = \int\limits_{-\infty}^{\infty}f(x)e^{-j\omega x}dx $

\subsection*{Gaussian}

\begin{eqnarray*}
  & & \mathcal{F}(e^{-x^2}) \\
  &=& \int\limits_{-\infty}^{\infty}e^{-x^2}e^{-j\omega x}dx \\
  &=& \int\limits_{-\infty}^{\infty}e^{-(x^2+j\omega x)}dx \\
  &=& \int\limits_{-\infty}^{\infty}e^{-\left(x^2+2x\frac{j\omega }{2}+\left(\frac{j\omega}{2}\right)^2-\left(\frac{j\omega}{2}\right)^2\right)}dx \\
  &=& \int\limits_{-\infty}^{\infty}e^{-\left(x + \frac{j\omega}{2}\right)^2}e^{\left(\frac{j\omega}{2}\right)^2}dx \\  
  &=& e^{\left(\frac{j\omega}{2}\right)^2}\int\limits_{-\infty}^{\infty}e^{-\left(x + \frac{j\omega}{2}\right)^2}dx \\
  &=& e^{-\frac{\omega^2}{4}}\int\limits_{-\infty}^{\infty}e^{-\left(x + \frac{j\omega}{2}\right)^2}dx \\  
  &=& e^{-\frac{\omega^2}{4}}\int\limits_{-\infty}^{\infty}e^{-\left( \frac{2x + j\omega}{2}\right)^2}dx \\ 
  &=& e^{-\frac{\omega^2}{4}}\frac{1}{2}\int\limits_{-\infty}^{\infty}e^{-\frac{(2x + j\omega)^2}{2(2)}}d(2x) \\
  &=& e^{-\frac{\omega^2}{4}}\frac{1}{2}(\sqrt{2\pi}\sqrt{2}) \\
  &=& e^{-\frac{\omega^2}{4}}\sqrt{\pi}
\end{eqnarray*}

\subsection*{Time scaling}
\begin{eqnarray*}
  & & \mathcal{F}(f(\frac{x}{a})) \\
  &=& \int\limits_{-\infty}^{\infty}{f(\frac{x}{a})e^{-j \omega x}dx} \\
  &=& a\int\limits_{-\infty}^{\infty}{f(\frac{x}{a})e^{-j\omega a\frac{x}{a}}d(\frac{x}{a})} \\
  &=& a\mathcal{F}(f)(a\omega)
\end{eqnarray*}

\subsection*{Differentiation}
\begin{eqnarray*}
  & & \mathcal{F}(f'(x)) \\
  &=& \int\limits_{-\infty}^{\infty}{f'(x)e^{-j \omega x}dx} \\
  &=& \int\limits_{-\infty}^{\infty}{e^{-j \omega x}df(x)} \\
  &=& f(x)e^{-j \omega x}|_{-\infty}^{\infty} - \int\limits_{-\infty}^{\infty}{f(x)de^{-j \omega x}} \\
  &=&  - \int\limits_{-\infty}^{\infty}{-j \omega f(x)e^{-j \omega x}dx} \\
  &=&  j\omega \mathcal{F}(x)
\end{eqnarray*}

\subsection*{Inverse}
\begin{eqnarray*}
  & & \int\limits_{-\infty}^{\infty}{\int\limits_{-\infty}^{\infty}{f(\tau)e^{-j\omega \tau}d\tau}e^{j\omega t}d\omega} \\
  &=& \int\limits_{-\infty}^{\infty}{f(\tau)\int\limits_{-\infty}^{\infty}{e^{-j\omega \tau}e^{j\omega t}d\omega}d\tau} \\
  &=& \int\limits_{-\infty}^{\infty}{f(\tau)\delta(t - \tau)d\tau} \\
  &=& f(t)
\end{eqnarray*}

\subsection*{Convolution}
\begin{eqnarray*}
  & & \mathcal{F}(f * g) \\
  &=& \int\limits_{-\infty}^{\infty}{\left(\int\limits_{-\infty}^{\infty}{f(\tau)g(t - \tau)d \tau}\right)e^{-j\omega t}dt} \\
  &=& \int\limits_{-\infty}^{\infty}{\left(\int\limits_{-\infty}^{\infty}{f(\tau)g(t - \tau)d \tau}\right)e^{-j\omega \tau}e^{-j\omega (t - \tau)}dt} \\
  &=& \int\limits_{-\infty}^{\infty}{\int\limits_{-\infty}^{\infty}{f(\tau)e^{-j\omega \tau}g(t - \tau)e^{-j\omega (t - \tau)}dt d\tau}} \\
  &=& \mathcal{F}(f) \mathcal{F}(g) \\
\end{eqnarray*}

\subsection*{A simple differential equation}
\begin{eqnarray*}
     \frac{dy}{dx} &=& Ay        \\
  \int\frac{dy}{y} &=& \int{Adx} \\
           \ln(y) &=& Ax + C     \\
                y &=& De^{Ax}    \\
             y(0) &=& D
\end{eqnarray*}

\section*{Solution}
Let $ \tilde{u}(t, \omega) = \mathcal{F}(u(t, x)) , \tilde{\phi}(\omega) = \mathcal{F}(\phi(x)) $

\begin{eqnarray*}
  \frac{\partial \tilde{u}(t, \omega)}{\partial t} &=& k(jw)^2\tilde{u}(t, \omega) \\
                                                   &=& -kw^2\tilde{u}(t, \omega)  \\
  \tilde{u}(t, \omega) &=& \tilde{\phi}(\omega)e^{-kw^2t}  \\
       u(t, x) &=& \mathcal{F}^{-1} (\tilde{\phi}(\omega)) * \mathcal{F}^{-1} (e^{-k \omega^2t}) \\
               &=& \frac{1}{\sqrt{4\pi kt}} \phi(x) * \mathcal{F}^{-1}\left(\sqrt{4kt}\sqrt{\pi}e^{-\frac{(\sqrt{4kt}\omega)^2}{4}}\right) \\
               &=& \frac{1}{\sqrt{4\pi kt}} \phi(x) * e^{-\left(\frac{x}{\sqrt{4kt}}\right)^2} \\
               &=& \frac{1}{\sqrt{4\pi kt}} \phi(x) * e^{-\frac{x^2}{4kt}} \\
\end{eqnarray*}

\section*{Conclusion}
Many of the derivations are not rigorous. That makes me wonder, what does it take to make sure this is rigorous. Turn out there are a lot of analytic background required for many of the operations. For example, check out the \href{https://en.wikipedia.org/wiki/Leibniz_integral_rule}{Leibniz integral rule} or \href{https://en.wikipedia.org/wiki/Distribution_(mathematics)}{Distribution}.

\end{document}
