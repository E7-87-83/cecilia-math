% arara: xelatex: { shell: yes, interaction: nonstopmode }
\documentclass{article}
\usepackage[UTF8]{ctex}

\title{嘗試理解 differential forms}
\author{Cecilia Chan}
\date{October 2022}

\begin{document}
\maketitle
在這篇文章裹,我想分享一下我對 differential form 的理解,大家有空可以看一下,有寫的不對的地方,歡迎指正。

\section{定義}
假設 f 是一個 $ \mathbf{R}^n \to \mathbf{R} $ 的函數,那假設我們有 $ p, v \in \mathbf{R}^n $, $ p $ 是一個點, $ v $ 是一個向量,我們就可以有這個數值

\begin{eqnarray*}
  \left. \frac{d}{dt} f(p + tv)\right|_{t = 0}
\end{eqnarray*}

明顯這個值和 $ p $ 和 $ v $ 相關,我們把它寫成 $ df_p(v) $,而且作這樣的理解。

在 $ p $ 這一點上,我們一個函數名字叫做 $ df_p $,它的輸入是一個向量 $ v $,它的輸出是一個實數。

一個輸入向量,輸出實數的函數有一個特殊的名字叫做 form,又可以叫做 linear functional 或者是 covector,這些都是同一東西的不同名字而已。

上面的定義和微分有關,出來的又是一個 form,就叫它 differential form。

在整個 $ \mathbf{R}^n $ 裹有無數個不同的點,在不同的點,有不同的 $ df_p $。正如我們可以有向量場 (vector field) 代表在不同地方有不同的向量,我們也可以用 form field 來代表在不同的地方有不同的 form。 $ df $ 就是一個 form field。

\section{一維例子}
最簡單的例子,假設 $ f: \mathbf{R} \to \mathbf{R} = \sin(x) $. 在這個一維的空間空間裏面,我們可以有一個向量 $ i $ 往右面走 1 步。這樣的話

\begin{eqnarray*}
  df_p &=& \left. \frac{d}{dt} f(p + tv)\right|_{t = 0} \\
       &=& \left. \frac{d}{dt} \sin(p + tki)\right|_{t = 0} \\
       &=& \left. k\cos(p + tki)\right|_{t = 0} \\
       &=& k \cos(p)
\end{eqnarray*}

所以在不同的點 $ p $ 上,$ df_p $ 是一個 form,它把 $ ki $ 帶到 $ k \cos(p) $。

\section{證明原題}
已知 $ \frac{dy}{dt} = \frac{f(t)}{h(y)} $,我們想象 $ y $ 和 $ t $ 都是 $ s $ 的函數,在任意點 $ p $

\begin{eqnarray*}
  dy_p(v) &=& \left. \frac{dy(p + tv)}{dt}\right|_{t = 0} \\
          &=& \left. \frac{dy(p + tki)}{dt}\right|_{t = 0} \\
          &=& k \left. \frac{dy}{ds}\right|_{s = p}
\end{eqnarray*}

同樣道理 $ dt_p(v) = k \left. \frac{dt}{ds}\right|_{s = p} $

所以
\begin{eqnarray*}
  \frac{dy}{dt} &=& \frac{f(t)}{h(y)} \\
  \frac{\frac{dy}{ds}}{\frac{dt}{ds}} &=& \frac{f(t)}{h(y)} \\
\end{eqnarray*}

(如果擔心 $ s $ 不存在,不妨想 $ s = t $,這樣就可以了)

相對於所有點 $ p $ 和所有 $ k \in R $,這樣都對,所以我們可以這樣寫

\begin{eqnarray*}
  h(y) k \left. \frac{dy}{ds}\right|_{s = p} &=& f(t) k \left. \frac{dt}{ds}\right|_{s = p} \\
  h(y) dy_p &=& f(t) dt_p \\
  h(y) dy &=& f(t) dt \\
\end{eqnarray*}

其中第一行到第二行是由兩個數值的相等,變成兩個 form 的相等,然後第二行到第三行是從兩個 form 的相等變成兩個 form field 的相等。

\end{document}