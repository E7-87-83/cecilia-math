\section*{Question 2}
\subsection*{Part a}
Consider the problem of finding the square root of $ T^* T = \left(\begin{array}{cc} 125 & 75 \\ 75 & 50 \end{array}\right) $ by assuming the square root is $ \left(\begin{array}{cc} x & y \\ y & z \end{array}\right)$. That gives us the equations:

\begin{eqnarray*}
  \left(\begin{array}{cc} x & y \\ y & z \end{array}\right)\left(\begin{array}{cc} x & y \\ y & z \end{array}\right) &=& \left(\begin{array}{cc} 125 & 75 \\ 75 & 50 \end{array}\right)
\end{eqnarray*}

or in an expanded form:
\begin{eqnarray*}
  x^2 + y^2 &=& 125 \\
    xy + yz &=& 75  \\
  y^2 + z^2 &=& 50
\end{eqnarray*}

These equations can be significantly simplified using Gröbner basis. We can find that basis using Sage easily as follows:

\begin{verbatim}
R.<x,y,z> = PolynomialRing(QQ, order='lex')
ideal(x^2 + y^2 - 125, x*y + y*z - 75, y^2 + z^2 - 50).groebner_basis()
\end{verbatim}

This command will give us the basis as:
\begin{eqnarray*}
  x + \frac{1}{10}z^3 - \frac{9}{2}z \\
  y + \frac{1}{10}z^3 - \frac{7}{2}z \\
  z^4 - 30z^2 + 125 
\end{eqnarray*}

Note that the last equation is has root $ \pm 5, \pm \sqrt{5} $ simply solving it as a quadratic equation, and once we know $ z $, it is trivial to compute $ x $ and $ y $. Therefore the roots are:

\begin{eqnarray*}
  x = 10, y = 5, z = 5 \\
  x = 10, y = -5, z = -5 \\
  x = 4\sqrt{5}, y = 3\sqrt{5}, z = 5 \\
  x = -4\sqrt{5}, y = -3\sqrt{5}, z = -5
\end{eqnarray*}

We simply pick the first one for simplicity. Once we know $ R $, we can simply compute $ S = TR^{-1} $ to find the full polar decomposition as :

\begin{eqnarray*}
  T &=& \left(\begin{array}{cc} 2 & -1 \\ 11 & 7 \end{array}\right) \\
    &=& \frac{1}{5}\left(\begin{array}{cc}3 & -4 \\ 4 & 3\end{array}\right)\left(\begin{array}{cc} 10 & 5 \\ 5 & 5 \end{array}\right)
\end{eqnarray*}

\subsection*{Part b}
We know that $ S $ is an isometry, so maximizing $ Tv $ is the same as maximizing $ Rv $, and $ R $ is much simpler. Consider $ v = \left(\begin{array}{c} x \\ y \end{array}\right)$

\begin{eqnarray*}
  |Tv|^2 &=& \left|\left(\begin{array}{cc} 10 & 5 \\ 5 & 5 \end{array}\right)\left(\begin{array}{c} x \\ y \end{array}\right)\right|^2 \\
       &=& \left|\left(\begin{array}{c} 10x + 5y \\ 5x + 5y \end{array}\right)\right|^2 \\
       &=& (10x + 5y)^2 + (5x + 5y)^2 \\
       &=& 125x^2 + 150xy + 50y^2 \\
\end{eqnarray*}

To optimize it, we will use the Lagrangian
\begin{eqnarray*}
  L &=& 125x^2 + 150xy + 50y^2 - \lambda(x^2 + y^2 - 1) \\
  \frac{\partial{L}}{\partial x} &=& 250x + 150y - 2\lambda x = 0 \\
  \frac{\partial{L}}{\partial y} &=& 150x + 100y - 2\lambda y = 0
\end{eqnarray*}

Using Gröbner basis again:
\begin{verbatim}
R.<x,y,z> = PolynomialRing(QQ, order='lex')
ideal(250*x + 150*y - 2*z*x,150*x + 100*y - 2*z*y,x^2+y^2-1).groebner_basis()
\end{verbatim}

We find the equations simplifies as follow:
\begin{eqnarray*}
  x - \frac{yz}{75} + \frac{2y}{3} &=& 0 \\
  y^2 + \frac{z}{375} - \frac{11}{15} &=& 0 \\
  z^2 - 175z + 625 &=& 0 
\end{eqnarray*}

And finally we solve as follows:
\begin{eqnarray*}
  z &=& \frac{25}{2}(7 \pm 3 \sqrt{5}) \\
  y &=& \pm\sqrt{\frac{1}{2}\left(1 \pm \frac{1}{\sqrt{5}}\right)} \\
  x &=& \pm\sqrt{\frac{1}{2}\left(1 \mp \frac{1}{\sqrt{5}}\right)} \\
\end{eqnarray*}

These four solutions correspond to two pairs of vectors, one pair that minimize $ |Tv| $, and one that maximize $ |Tv| $. The solution that maximize $ |Tv| $ is:

\begin{eqnarray*}
  x &=& \sqrt{\frac{1}{2}\left(1 + \frac{1}{\sqrt{5}}\right)} \\
  y &=& \sqrt{\frac{1}{2}\left(1 - \frac{1}{\sqrt{5}}\right)}
\end{eqnarray*}

Last but not least, the maximum length is 
\begin{eqnarray*}
  & & \sqrt{125x^2 + 150xy + 50y^2} \\
  &=& \sqrt{\frac{25}{2} (7 + 3 \sqrt{5})} \\
  &=& 5\sqrt{\frac{7 + 3\sqrt{5}}{2}} \\
  &=& 5\left(\frac{3 + \sqrt{5}}{2}\right)
\end{eqnarray*}