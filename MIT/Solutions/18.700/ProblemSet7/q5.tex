\section*{Question 5}
\subsection*{Part a}
Following the last question, we need to make sure $ \langle (1, 2, 3), (2, 5, a) \rangle = 0 $, that gives $ a = -4 $.

\subsection*{Part b}
Since $ \langle (1, 2, 3), (2, 5, a) \rangle = 0 $, there should be a third vector that is orthogonal to both of them. Once we find it, this can be used as a basis.

We can find the third vector using the reduced echelon form:
\begin{eqnarray*}
  \left(\begin{array}{ccc}
    1 & 2 & 3 \\
    2 & 5 & -4
  \end{array}\right) \to
  \left(\begin{array}{ccc}
    1 & 0 &  23 \\
    0 & 1 & -10
  \end{array}\right) 
\end{eqnarray*}

So we can easily pick $ (-23, 10, 1) $ as the last vector.

We claim that this vector is actually an eigenvector. To show that, let's consider the matrix of the mapping in this basis

\begin{eqnarray*}
\left(\begin{array}{ccc}
0 & 0 & ? \\
0 & 1 & ? \\
0 & 0 & ?
\end{array}\right)
\end{eqnarray*}

By part 1, we know that the matrix of its adjoint should be.

\begin{eqnarray*}
\left(\begin{array}{ccc}
0 & 0 & 0 \\
0 & 1 & 0 \\
? & ? & ?
\end{array}\right)
\end{eqnarray*}

But the operator is self-adjoint, that means the matrix must be:

\begin{eqnarray*}
\left(\begin{array}{ccc}
0 & 0 & 0 \\
0 & 1 & 0 \\
0 & 0 & ?
\end{array}\right)
\end{eqnarray*}

Whatever the last entry is, it must be the case that $ (-23, 10, 1) $ is an eigenvector.
