\section*{Question 3}
Consider a general square root of $ I \in \mathbf{F_q}^2 $ as follow:
\begin{eqnarray*}
  \left(\begin{array}{cc}
    a & b \\ 
    c & d 
  \end{array}\right)\left(\begin{array}{cc}
    a & b \\ 
    c & d 
  \end{array}\right) &=& \left(\begin{array}{cc}
    1 & 0 \\ 
    0 & 1 
  \end{array}\right)
\end{eqnarray*}
This boils down to the equations
\begin{eqnarray*}
  a^2 + bc = bc + d^2 = 1 \\
  b(a + d) = c(a + d) = 0 
\end{eqnarray*}
The latter equation is easier to solve, it is either $ b = c = 0 $ or $ a = -d $ (or both).

In case $ b = c = 0 $, we have $ a^2 = d^2 = 1 $, so $ a = \pm 1 $ and $ d = \pm 1 $, signs can be chosen arbitrarily.

In case $ a = -d $, $ bc = 1- a^2 $. So it depends on the value of $ 1 - a^2 $. Suppose $ 1 - a^2 = 0 $, either $ b $ or $ c $ must be 0, and the other is arbitrary. Otherwise, $ 1 - a^2 \ne 0 $, we can have an arbitrary non-zero $ b $, and $ c $ has no choice but to be $ b^{-1} (1 - a^2)$

\subsection*{Case 1}
In the case the field $ F $ has characteristic 2. $ a = -a $ for all field elements. So we have exactly 1 solution $ a = d = 1 = -1 $ when $ b = c = 0 $.

The only case for $ 1 - a^2 = 0 $ is when $ a = 1 $. Let's say $ b = 0 $, $ c $ have $ q $ choices, but we must avoid $ c = 0 $ to avoid double counting, that is $ q - 1 $ solutions. Symmetrically, we have $ q - 1 $ solution for $ c = 0 $ case. Together we have $ 2(q - 1) $ solutions.

Last but not least, there are $ q - 1 $ possibility of $ a $ such that $ 1 - a^2 \ne 0 $. For each of these case, $ b $ can be an arbitrary non-zero which has $ q - 1 $ cases, and $ c $ has no choice, so the number of solution of this type is $ (q - 1)^2 $.

Therefore the total number of solutions should be $ 1 + 2(q - 1) + (q - 1)^2 = q^2 $

\subsection*{Case 2}
Otherwise, the field $ F $ does not have characteristic 2, $ a \ne -a $, and therefore we have exactly 4 solution $ a = d = \pm 1 $ when $ b = c = 0 $.

There are two cases $ 1 - a^2 = 0 $ when $ a = \pm 1 $. Let's say $ b = 0 $, $ c $ have $ q $ choices, but we must avoid $ c = 0 $ to avoid double counting, that is $ 2(q - 1) $ solutions. Symmetrically, we have $ 2(q - 1) $ solution for $ c = 0 $ case. Together we have $ 4(q - 1) $ solutions.

Last but not least, there are $ q - 2 $ possibility of $ a $ such that $ 1 - a^2 \ne 0 $. For each of these case, $ b $ can be an arbitrary non-zero which has $ q - 1 $ cases, and $ c $ has no choice, so the number of solution of this type is $ (q - 2)(q - 1) $.

Therefore the total number of solutions should be $ 4 + 4(q - 1) + (q - 2)(q - 1) = q^2 + q + 2 $

The solution is validated using this simple Sage script:
\begin{verbatim}
Field = GF(3)
MatrixOnField = MatrixSpace(Field,2,2)
elements = []
for _,element in enumerate(Field):
  elements.append(element)
count = 0
for a in elements:
  for b in elements:
    for c in elements:
      for d in elements:
        A = MatrixOnField([[a,b],[c,d]])
        if (A * A == MatrixOnField([[1,0],[0,1]])):
          count = count + 1
print(count)
\end{verbatim}