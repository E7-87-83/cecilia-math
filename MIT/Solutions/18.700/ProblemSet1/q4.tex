\section*{Question 4}
Note that $ F^3 $ stands for either $ R^3 $ or $ C^3 $ in the book.
\subsection*{Part a}
It is a vector space because:
\begin{enumerate}
    \item {$ (0, 0, 0) $ is in the subset}
    \item { If $ a + 2b + 3c = 0 $ and $ d + 2e + 3f = 0 $, then $ (a + d) + 2(b + e) + 3 (c + f) = 0 $. Therefore, whenever $ (a,b,c) $ and $ (d,e,f) $ belongs to the subset, then $ (a,b,c) + (d,e,f) = ((a+d),(b+e),(c+f)) $ belongs to the subset too, which means the space is closed under addition.}
    \item { If $ a + 2b + 3c = 0 $ then $ (fa) + 2(fb) + 3(fc) = 0 $. Therefore, whenever $ (a,b,c) $ belongs to the subset, then $ f(a, b, c) = (fa, fb, fc) $ belongs to the subset too, which means the space is closed under scalar multiplication.}
\end{enumerate}

\subsection*{Part b}
The vector $ (0, 0, 0) $ does not belong to the subset and therefore it is not a subspace.

\subsection*{Part c}
The vector $ (1, 1, 0) $ and $ (0, 0, 1) $ belongs to the subset but their sum $ (1, 1, 1) $ is not, therefore it is not a subspace.

\subsection*{Part d}
It is a vector space because:
\begin{enumerate}
    \item {$ (0, 0, 0) $ is in the subset}
    \item { If $ a = 5c $ and $ d = 5f $, then $ (a + d) = 5(c + f) $. Therefore, whenever $ (a,b,c) $ and $ (d,e,f) $ belongs to the subset, then $ (a,b,c) + (d,e,f) = ((a+d),(b+e),(c+f)) $ belongs to the subset too, which means the space is closed under addition.}
    \item { If $ a = 5c $ then $ (fa) = 5(fc) $. Therefore, whenever $ (a,b,c) $ belongs to the subset, then $ f(a, b, c) = (fa, fb, fc) $ belongs to the subset too, which means the space is closed under scalar multiplication.}
\end{enumerate}

Apparently, this proof is the same as part a. In fact, it follows that a set of vectors satisfying a system of homogeneous linear equations (i.e. with 0 constant terms) is a subspace, and the proof is going to be identical. 