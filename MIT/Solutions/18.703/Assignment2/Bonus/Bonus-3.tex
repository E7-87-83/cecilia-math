\section*{Bonus-3}
Finding the Bezout's identity can be done by the Extended Euclidean algorithm. The observation is that we can build an equation if we have the equation of the subproblem.

Let's recall the Euclidean algorithm. Given $ a, b \in \mathbb{Z} $, we want to find the GCD, here is how we do it.

\begin{verbatim}
def gcd(a, b):
    if a < 0:
        return gcd(-a, b)
    if b < 0:
        return gcd(a, -b)
    if a < b:
        return gcd(b, a)
    if b == 0:
        return a
    return gcd(b, a % b)
\end{verbatim}

Imagine now we want to extend the algorithm so that it also give us the Bezout's identity. The cases are relative straight forward, we just need to add a few lines.

\begin{verbatim}
def gcd(a, b):
    if a < 0:
        # (m)(-a) + (n)(b) = g => (-m)(a) + (n)(b) = g
        (m, n, g) = gcd(-a, b)
        return (-m, n, g)
    if b < 0:
        # (m)(a) + (n)(-b) = g => (m)(a) + (-n)(b) = g
        (m, n, g) = gcd(a, -b)
        return (m, -n, g)
    if a < b:
        # (m)(b) + (n)(a) = g => (n)(a) + (m)(b) = g
        (m, n, g) = gcd(b, a)
        return (m, m, g)
    if b == 0:
        # (1)(a) + (0)(b) = a
        return (1, 0, a)
    else:
        # a = bq + r
        # (m)(b) + (n)(r) = g
        # => (m)(b) + (n)(a - bq) = g
        # => (n)(a) + (m-nq)(b) = g
        (m, n, g) = gcd(b, a % b)
        return (n, m - n * (a // b), g)
\end{verbatim}

With the Bezout's identity, we can find multiplicative inverses, and then we can find the $ k $ of the Chinese Remainder Theorem.