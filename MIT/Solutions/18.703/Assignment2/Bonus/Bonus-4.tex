\section*{Bonus-4}
This given is simply not true. Two identical cycle obviously intersects and commutes. Here I will investigate the conditions for two cycles to commute.

Cycles with length 1 is simply the identity function and always commute.

With proper renaming of elements, we can always make $ \tau $ to be the cycle $ (1, 2, \cdots, n ) $. 

To investigate commutativity, we will use this relationship:

$ \sigma \tau = \tau \sigma \iff \sigma \tau \sigma^{-1} = \tau $

Note that $ \sigma \tau \sigma^{-1} (\sigma(1)) = \sigma \tau 1 = \sigma (2) $, we can easily see that $ \sigma \tau \sigma^{-1} = (\sigma(1), \sigma(2), \cdots, \sigma(n)) $.

In order for $ \sigma \tau \sigma^{-1} = \tau $, we must have $ \sigma(i) = (i + k) \pmod n $. That's a pretty severe restriction on $ \sigma $.

One obvious case is that if the $ \sigma $'s cycle, as a set, is not exactly $ \{1, 2, \cdots, n \} $, then it is impossible for $ \sigma $ and $ \tau $ to commute.