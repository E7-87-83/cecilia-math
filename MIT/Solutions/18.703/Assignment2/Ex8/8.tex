\section*{Ex8 - 8}
Now that I have a program, I can obviously just loop it 100 times, but that feels like cheating. Another way of doing the same thing is to factor it into disjoint cycles. Once I get the disjoint cycles we know they commute and the power will be trivial to find. Here is the procedure to factor a permutation into disjoint cycles.

\begin{verbatim}
def cycles(perm):
    included = {}
    cycles = []
    for p in perm:
        if p not in included:
            included[p] = True
            cycle = [p]
            current = p
            while True:
                next = perm[current - 1]
                if next == p:
                    break
                else:
                    included[next] = True
                    cycle.append(next)
                    current = next
            cycles.append(cycle)
    return cycles
\end{verbatim}

Turn out $ \sigma $ is simply a single cycle of length 6 $ ([3 4 5 6 2 1) $, which means if it is applied 6 times, it will be the identity. So $ \sigma^{100} = \sigma^4 $, and the latter can be easily found by a simple loop.

The answer is $ \left(\begin{array}{cccccc}1 & 2 & 3 & 4 & 5 & 6\\6 & 5 & 2 & 1 & 3 & 4\end{array}\right) $