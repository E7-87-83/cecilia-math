\section*{Ex4 - 20}
To begin with, here is table 4.22 with empty cells annotated with capital letters so that I can refer to them later.

\begin{tabular}{ c|c|c|c|c }
      & e & a & b & c \\
    \hline
    e & e & a & b & c \\
    a & a & P & Q & R \\
    b & b & S & T & U \\
    c & c & V & W & X 
\end{tabular}

Note that in order to have multiplicative inverse, each row and each column must not have duplicated elements, we will use this rule repeatedly.

Following the problem statement, we will first try set $ P = e $. That leave us with $ \{Q, R\} = \{b, c\} $, but then $ Q \ne b $ because that would create a duplicate in the $ b $ column. So $ Q = c $, $ R = b $. Similar reasoning leads to $ S = c $, $ V = b $. Now the table becomes this.

\begin{tabular}{ c|c|c|c|c }
    & e & a & b & c \\
  \hline
  e & e & a & b & c \\
  a & a & e & c & b \\
  b & b & c & T & U \\
  c & c & b & W & X 
\end{tabular}

Now we could have either $ T = a $ or $ T = e $, this leads to these complete tables $ G1 $ and $ G2 $.

G1

\begin{tabular}{ c|c|c|c|c }
    & e & a & b & c \\
  \hline
  e & e & a & b & c \\
  a & a & e & c & b \\
  b & b & c & a & e \\
  c & c & b & e & a 
\end{tabular}

G2

\begin{tabular}{ c|c|c|c|c }
    & e & a & b & c \\
  \hline
  e & e & a & b & c \\
  a & a & e & c & b \\
  b & b & c & e & a \\
  c & c & b & a & e 
\end{tabular}

Alternatively, we could have set $ P = b $ instead, following the same logic, we will reach this table $ G3 $.

G3

\begin{tabular}{ c|c|c|c|c }
    & e & a & b & c \\
  \hline
  e & e & a & b & c \\
  a & a & b & c & e \\
  b & b & c & e & a \\
  c & c & e & a & b 
\end{tabular}

Note that $ G2 $ has a unique property that every element is an inverse of itself, so it cannot be isomorphic with $ G1 $ or $ G3 $. On the other hand, if we swap $ a $ and $ b $ in $ G1 $, then it becomes $ G3 $, so $ G1 $ and $ G3 $ are isomorphic.

Note that at this point we only listed these are the only possible groups, we haven't verified for associativity yet.

\subsection*{Part a}
Yes, all groups of order 4 are commutative.

\subsection*{Part b}
$ G3 $ is isomorphic to $ U4 $ by mapping $ e, a, b, c $ to $ 1, i, -1, -i $ respectively. We also know $ G1 $ is isomorphic to $ G3 $, so $ G1 $ is also isomorphic to $ U4 $.

\subsection*{Part c}
$ G2 $ is isomorphic to the group of diagonal matrix of size $ 2 \times 2 $ with $ 1 $ and $ -1 $ only on the diagonal as follows:

\begin{eqnarray*}
    e = \left(\begin{array}{cc} 1 & 0 \\ 0 & 1 \end{array}\right) \\
    a = \left(\begin{array}{cc} -1 & 0 \\ 0 & 1 \end{array}\right) \\
    b = \left(\begin{array}{cc} 1 & 0 \\ 0 & -1 \end{array}\right) \\
    c = \left(\begin{array}{cc} -1 & 0 \\ 0 & -1 \end{array}\right)
\end{eqnarray*}

Now we have proved that there are two non-isomorphic groups of order 4. Both of them are Abelian.