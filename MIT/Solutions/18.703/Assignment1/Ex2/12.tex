\section*{Ex2 - 12}
\begin{enumerate}
\item{If $ S $ has only one element, say, $ e $, we can only define $ e * e = e $, so there is only one binary operations.}
\item{If $ S $ has only two elements, the multiplication table have 4 cells, without any further restriction, each of these cells has 2 options, therefore we have $ 2^4 = 16 $ possible binary operations.}
\item{If $ S $ has only three elements, the multiplication table have 9 cells, without any further restriction, each of these cells has 2 options, therefore we have $ 3^9 = 19683 $ possible binary operations.}
\item{If $ S $ has only $ n $ elements, the multiplication table have $ n^2 $ cells, without any further restriction, each of these cells has 2 options, therefore we have $ n^(n^2) $ possible binary operations.}
\end{enumerate}

Just as a comment - there could be $ 4^{16} = 4294967296 $ binary operations on 4 elements, but there are only 2 non-isomorphic groups of order 4.