\section*{Problem 10}
\subsection*{Part a}
\begin{center}
  \begin{tabular}{ |c|c|c| }
    \hline
    Score             & Class mark & Frequency \\
    \hline
    $ 44 \le x < 52 $ & 48         & 3         \\
    $ 52 \le x < 60 $ & 56         & 9         \\
    $ 60 \le x < 68 $ & 64         & 15        \\
    $ 68 \le x < 76 $ & 72         & 11        \\
    $ 76 \le x < 84 $ & 80         & 2         \\
    \hline
  \end{tabular}
\end{center}
\subsection*{Part b}
The mean is $ (48 \times 3 + 56 \times 9 + 64 \times 15 + 72 \times 11 + 80 \times 2) / 40 = 64 $.

The standard deviation is $ \sqrt{ \frac{1}{40} \times (3 \times (48 - 64)^2 + 9 \times (56 - 64)^2 + 15 \times (64 - 64)^2 + 11 \times (72 - 64)^2 + 2 \times (80 - 64)^2) } = 8 $.

Note: The entire population is given (i.e. the 40 students), therefore the standard deviation is calculated using the population formula. We are estimating only because we are given the histogram, not the accurate marks.

\subsection*{Part c}
The standard score is given by the formula $ \frac{x - \mu}{\sigma} $, where $ x $ is the score, $ \mu $ is the mean and $ \sigma $ is the standard deviation. Therefore the standard score for Susan is $ \frac{76 - 64}{8} = 1.5 $

\subsection*{Part d}
Let her new score be $ x $, then her new standard score would be $ \frac{x - 58}{10} $. Since she is performing equally well, she would have the same standard score 1.5, therefore:

\begin{eqnarray*}
  \frac{x - 58}{10} &=& 1.5 \\
             x - 58 &=& 15  \\
                 x &=& 73
\end{eqnarray*}

Therefore her estimated score in the second test is 73.

