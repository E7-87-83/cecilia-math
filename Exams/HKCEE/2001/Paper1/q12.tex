\section*{Problem 12}
\subsection*{Part a}
\subsubsection*{Part i}
The perimeter of $ F_{40} $ is 10cm increased by 1cm 39 times, therefore the perimeter is $ 10 + 39 = 49 $cm.

\subsubsection*{Part ii}
The sum of the perimeters is $ 10 + 11 + \cdots + 49 $, which is an arithmetic progression with $ a = 10 $, $ d = 1 $ and $ n = 40 $. Therefore the sum is $ \frac{40}{2} \times (10 + 49) = 20 \times 59 = 1180 $cm.

\subsection*{Part b}
\subsubsection*{Part i}
The ratio of areas is the square of the ratio of the perimeters, therefore
\begin{eqnarray*}
    \frac{\text{Area}(F_2)}{\text{Area}(F_1)} &=& \left(\frac{11}{10}\right)^2 \\
    \frac{\text{Area}(F_2)}{4} &=& \left(\frac{11}{10}\right)^2 \\
    \text{Area}(F_2) &=& 4 \times \left(\frac{11}{10}\right)^2 \\
                     &=& 4 \times \frac{121}{100} \\
                     &=& 4.84
\end{eqnarray*}
\subsubsection*{Part ii}
Using the same argument, we can compute the area of $ F_3 $ as follows:
\begin{eqnarray*}
    \frac{\text{Area}(F_3)}{\text{Area}(F_1)} &=& \left(\frac{12}{10}\right)^2 \\
    \frac{\text{Area}(F_3)}{4} &=& \left(\frac{12}{10}\right)^2 \\
    \text{Area}(F_3) &=& 4 \times \left(\frac{12}{10}\right)^2 \\
                     &=& 4 \times \frac{144}{100} \\
                     &=& 5.76
\end{eqnarray*}
Since $ \text{Area}(F_3) - \text{Area}(F_2) = 0.92 $ but $ \text{Area}(F_2) - \text{Area}(F_1) = 0.84 $, it is clear that the areas does not form an arithmetic sequence.

