\section*{Question 1}
\subsection*{Part a}
By the first and the third equation, we have
\begin{eqnarray*}
  (k + 2) x + 6z = 0
\end{eqnarray*}
This line must be parallel to $ x + (k + 1)z = 0 $, because otherwise we will have only one solution at the intersection point. So $ (k+2)x + (k+2)(k+1)z \equiv (k+2)x + 6z $. 

Comparing coefficients, we have $ (k+2)(k+1) = 6 $. 

Solving, get $ k = 1 $ or $ k = -4 $.

\subsection*{Part b}
In case $ k = 1 $, using the second equation, we have $ x + 2z = 0 $. So $ x = -2z $. Using the first equation, we have $ 2(-2z) + y + 2z = 0 $, so $ y = 2z $. Therefore the answer is $ (-2z, 2z, z) $ when $ k = 1$.

In case $ k = -4 $, using the second equation, we have $ x - 2z = 0 $. So $ x = 2z $. Using the first equation, we have $ 2(2z) + y + 2z = 0 $, so $ y = -4z $. Therefore the answer is $ (2z, -4z, z) $ when $ k = -1$.