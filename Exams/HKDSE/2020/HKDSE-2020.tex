\documentclass{article}
\usepackage[utf8]{inputenc}

\title{DSE}
\author{Cecilia}
\date{April 2021}

\begin{document}

\maketitle

% https://dse.life/static/pp/m0/eng/dse/2020/p1.pdf

\section{Q1}
\begin{eqnarray*}
  & & \frac{(mn^{-2})^5}{m^{-4}} \\
  &=& (mn^{-2})^5m^4 \\
  &=& (m^5n^{-10})m^4 \\
  &=& (m^9n^{-10}) \\
  &=& \frac{m^9}{n^{10}}
\end{eqnarray*}

\section{Q2}
\subsection{Q2a}
\begin{eqnarray*}
  & & \alpha^2 + \alpha - 6 \\
  &=& (\alpha + 3)(\alpha - 2)
\end{eqnarray*}

\subsection{Q2b}
\begin{eqnarray*}
  & & \alpha^4 + \alpha^3 - 6\alpha^2 \\
  &=& \alpha^2(\alpha^2 + \alpha - 6) \\
  &=& \alpha^2 (\alpha + 3)(\alpha - 2)
\end{eqnarray*}

\section{Q3}
\subsection{Q3a}
534.77
\subsection{Q3b}
534.77
\subsection{Q3c}
530

\section{Q4}
We know:
\begin{eqnarray*}
  \frac{a}{b} &=& \frac{6}{7} \\
  \frac{b}{a} &=& \frac{7}{6} \\
\end{eqnarray*}
and
\begin{eqnarray*}
           3a &=& 4c \\
  \frac{c}{a} &=& \frac{3}{4} \\
\end{eqnarray*}
Therefore:
\begin{eqnarray*}
   & & \frac{b + 2c}{a + 2b} \\
   &=& \frac{\frac{b}{a} + 2\frac{c}{a}}{\frac{a}{a} + 2\frac{2b}{a}} \\
   &=& \frac{\frac{7}{6} + 2 \times \frac{3}{4}}{1 + 2 \times \frac{7}{6}} \\ 
   &=& \frac{4}{5} % check : ((7/6) + 2*(3/4))/(1 + 2 * 7/6)
\end{eqnarray*}

\section{Q5}
Let the number of male applicant be $ m $ and the number of female applicant be $ f $.
\begin{eqnarray*}
          m &=& 1.28f \\
      m - f &=& 91 \\
  1.28f - f &=& 91 \\
      0.28f &=& 91 \\
          f &=& 325 \\
          m &=& 1.28 \times 325 \\
           &=& 416 
\end{eqnarray*}
Therefore the number of male applicant is 416.

\section{Q6}
\subsection{Q6a}
\begin{eqnarray*}
  3 - x &>& \frac{7-x}{2} \\
  6 - 2x &>& 7 - x \\
  -1 &>& x \\
  x &<& -1
\end{eqnarray*}

\begin{eqnarray*}
  5 + x &>& 4 \\
  x &>& -1
\end{eqnarray*}

Therefore the solution is $ x > -1 $ or $ x < -1$, or equivalently $ x \ne -1$.

\subsection{Q6b}
Since $ x $ cannot be -1, therefore the greatest negative integer satisfying (*) is -2.

\section{Q7}
\subsection{Q7a}
If $ p(x) = 0 $ has equal roots, $ \delta = 0$. Therefore $ 12^2 - 4(4)(c) = 0 $, we get $ c = 9 $

\subsection{Q7b}
\begin{eqnarray*}
  & & p(x) - 169 \\
  &=& 4x^2 + 12x + 9 - 169 \\
  &=& 4x^2 + 12x - 160 \\
  &=& 4(x^2 + 3x - 40) \\
  &=& 4(x + 8)(x - 5) \\
\end{eqnarray*}
Therefore the x-intercepts are $ (-8, 0) $ and $ (5, 0) $

\section{Q8}
\subsection{Q8a}
Because $ \delta BEA $ is an isosceles triangle, we have angle $ BEA = 30 $. \\
Angle $ <BEA + <EBD = <ABD $, therefore, $ <EBD = 42 - 30 = 12 $. \\
$ <BEC = < EBD = 12 $ because $ BD $ parallel $ CE $. \\

\subsection{Q8b}
$ < CFE + <FEC + <ECF = 180 $ because it is a triangle. \\
$ <ECF = <BDC = \theta $ because $ BD $ parallel $ CE$. \\
Therefore $ < CFE = 180 - 12 - \theta = 168 - \theta $.

\end{document}