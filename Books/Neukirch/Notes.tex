\documentclass{article}
\usepackage[utf8]{inputenc}
\usepackage{amsmath}

\title{Neukirch Reading Notes}
\author{Cecilia}
\date{June 2021}
\DeclareMathOperator{\fractionfield}{frac}
\DeclareMathOperator{\closure}{closure}
\begin{document}
\maketitle

\section*{Page 7}
After we obtained the relation $ \det(bE - (a_{ij}))\omega_i = 0 $, we knew $ \omega_i \ne 0 $ because it is in the generating set, but we cannot conclude $ \det(bE - (a_{ij}))\omega_i = 0 $ because of the possibility of zero divisor. (Recall $ A[b_1 \cdots b_n] $ is just a ring, no requirement for it to be an integral domain).

Since $ \omega_i $ is a generating set, we can use it to generate 1. $ 1 = \sum c_i \omega_i $. We are sure $ 1 \ne 0 $ because $ A[b_1 \cdots b_n ] $ cannot be the zero ring. Now multiply both sides by $ \det(bE - (a_{ij})) $. Now either $ \det(bE - (a_{ij})) = 0 $ so that both side is 0, or it isn't so that the left hand side is not zero but the right hand side is zero. That concludes $ \det(bE - (a_{ij})) = 0 $

In the last paragraph, it was written it is immediate ..., that is not obvious at all, especially to someone who is not familiar with the terminologies just introduced.

The goal is to prove that the "integral closure" of $ A $ over $ B $ is "integrally closed" in $ B $. Proposition 2.4 involves the use of "integral over". So the whole proof is really a soup of all these terms.

To make it easier to understand, instead of using just an overbar, I write out the closure explicitly. $ \closure(P|Q) $ means the integral closure of $ P $ over $ Q $.

There is a very easy lemma that we can start with:

$ \closure(A | B) $ is integral over $ A $.

This is really just reading the definitions. The closure of $ A $ over $ B $ is all the elements of $ b $ that is integral over $ A $, which is required for the definition of integral over.

$ \closure(A | B) \subseteq B $.

This is just as easy - the definition of closure requires the elements comes from $ B $.

Now, let's examine the statement we wanted to prove. $ \closure(A | B) $ is integrally closed in $ B $, that means:

$ \closure(\closure(A | B) | B) = \closure(A | B) $.

To prevent a notation soup, let $ C = \closure(A | B) $ and $ D = \closure(C | B) $, now our goal becomes proving $ C = D $.

Using our lemmas, we see $ C $ is integral over $ A $, $ C \subseteq B $, $ D $ is integral over $ C $, $ D \subseteq B $. Invoking proposition 2.4, we see $ D $ is integral over $ A $.

Now we want to argue set equality, to do that, we see 
$ C \subseteq D $ because elements in $ C $ must be integral in $ C $ so they are contained in $ D $ as the integral closure of $ C $ over $ B $.

$ D \subseteq C $ because elements in $ D $ must be integral over $ A $, but $ C $ contains all such elements.

That concludes the proof of $ C = D $ and therefore the statement that the integral closure of $ A $ over $ B $ is integrally closed in $ B $.

Not intermediate at all!

\section*{Page 8}
To prove $ B $ is integrally closed, we need to prove

$ \closure(B|\fractionfield(B)) = B $

Let $ C = \closure(B|\fractionfield(B)) $, now $ C $ is integral over $ B $, $ B $ is integral over $ A $ because $ B = \closure(A|L) $. Proposition 2.4 gives $ C $ is integral over $ A $. As above, we claim $ B \subseteq C $ and then $ C \subseteq B $ because elements in $ C$ must be integral over $ A $ but $ B $ contains all such elements.

As $ L | K $ is a finite extension, the set of vectors $ 1, \beta, \beta^2, \cdots, \beta^n $ must be linearly dependent, and that why we can claim the polynomial is 0.

The "hence the same holds for all coefficients" is tricky. The coefficients are sum and products of the roots, so if the roots are integral, so are the coefficients.

\end{document}
