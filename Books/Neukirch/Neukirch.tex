\documentclass{article}
\usepackage[utf8]{inputenc}
\usepackage{amsmath}

\title{Neukirch}
\author{Cecilia}
\date{June 2021}

\begin{document}
\maketitle

\section*{Exercise 1}
If $ \alpha $ is a unit, then $ N(\alpha) N(\alpha^{-1}) = N(1) = 1 $, but $ N $ is positive integer, so $ N(\alpha) = 1 $.

If $ N(\alpha) = 1 $, then $ N(\overline{\alpha}) = 1 $ and $ N(\alpha \overline{\alpha}) = 1 $. But $ \alpha \overline{\alpha}$ is real so $ \alpha \overline{\alpha} = 1 $ and $ \overline{\alpha} $ is a multiplicative inverse of $ \alpha $, $ \alpha $ is a unit.

\section*{Exercise 2}
$ Z[i] $ is UFD, so we can prime factorize both side.

The right hand side consists of a unit times a pile of factors, each repeated n times.

The left hand side consist of two groups, with some prime power factors comes from $ \alpha $ and some prime powers comes from $ \beta $.

Now, we can do the matching, the key is that one factor belong to exactly one of $ \alpha $ or $ \beta $, and they always get repeated $ n $ times, that is why we can express the answer in the required form.

\section*{Exercise 3}

$ \gcd((x + yi), (x - yi)) = \gcd((x + yi), 2x) = 1 $ since $ \gcd(x, y) = 1 $.

By Exercise 2, $ (x + yi)(x - yi) = z^2 => x + yi = e \alpha^2 $

\begin{eqnarray*}
  x + yi &=& e \alpha^2   \\
         &=& e (u + vi)^2 \\
         &=& e (u^2 + 2uvi -v^2)
\end{eqnarray*}

Units of Gaussian integers are $ 1, -1, i, -i, $ so changing $ e $ only permute the solution.

\section*{Exercise 4}
Recall the ordered field axioms:

\begin{enumerate}
    \item if $ a < b $ then $ a + c < b + c $, and 
    \item if $ 0 < a $ and $ 0 < b $ then $ 0 < ab $.
\end{enumerate}

If $ 0 < i $, then $ 0 < -1 = i^2 $ (by axiom 2), then $ 0 < 1 = (-1)^2 $ (by axiom 2 again)
but since $ 0 < -1 $, then $ 1 = 0 + 1 < -1 + 1 = 0 $ (by axiom 1) contradicts with above.

If $ i < 0 $, then $ 0 = i - i < 0 - i = -i $ (by axiom 1), then $ 0 < -1 = (-i)^2 $ (by axiom 2), then $ 0 < 1 = (-1)^2 $ (by axiom 2 again)
but since $ 0 < -1 $, then $ 1 = 0 + 1 < -1 + 1 = 0 $ (by axiom 1) contradicts with above.

Therefore the field cannot be ordered.

\section*{Exercise 5}

Define $ N(\alpha) = |\alpha|^2 $, $ N $ is multiplicative, therefore $ N(\alpha) = 1 $ if $ \alpha $ is a unit. Now $ d > 1 $, so $ (a + b\sqrt(-d))(a - b\sqrt(-d)) = a^2 + b^2d $ , so $ b $ must be 0, $ a $ must be 1 or -1.

\section*{Exercise 6}

The theory of Pell's equation gives us infinite solutions for the Diophantine equation $ x^2 - dy^2 = 1 $.

Now for every such solution $ (x + y \sqrt{d})(x - y \sqrt{d}) = x^2 - d^2y = 1 $, therefore $ (x \pm y \sqrt{d}) $ is a unit. In other words, we have infinitely many units.

\section*{Exercise 7}
Qwq

\end{document}
